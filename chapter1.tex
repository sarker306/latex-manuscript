\section{সুচনা}
সে বাসায় ফেরে, প্রতিদিন অনেক রাত হয় তার ফিরতে। বাসায় ফিরে বলে, "কে কোথায়?"

কেউ কোন কথা বলে না। পরীর মত ছোট্ট মেয়েটি, যাকে মাথায় চড়িয়ে সারাঘর ঘুরে বেড়ানো ছিলো তার প্রতিদিনের রুটিনের অংশ, সে ডুবে থাকে রোজকার হোমওয়ার্কের সাগরে। একবার হয়তো সে বের হয়, রাতে খাবারের টেবিলে দেখা হয় তার সাথে, হয়তো তাও হয় না।

পিচ্চি ছেলেটি, আজ তার বন্ধুর জন্মদিন অথবা এমন কিছু। বাসায় নেই, ফিরতে রাত করবে। কতদিন যে এই গুটলুটার সাথে কথা হয় না! বয়ঃসন্ধির কারণেই কি না, গুটিয়ে গেছে গুটলু ভীষণ। যতক্ষণ বাসায় থাকে, কম্পিউটারটা থেকে নড়ে না। প্লেস্টেশনটাও তার জোরাজুরিতে ড্রইংরুম থেকে ওর রুমে সরাতে হয়েছে। এই তো মাত্র এক বছর আগেও সে বাসায় লোকজন আসলে হৈ হৈ করে দৌড়ে বেড়াতো, গেস্টের কাছে বসে বকরবকর করতো, আজ মনে হয় সেগুলো না জানি কত দূর অতীতের কাহিনী।

একটু আরাম করার জন্যে সে বসলো ড্রইংরুমে। স্ত্রী এসে এক কাপ চা দিয়ে গেলো তাকে। কোন কথা নেই। কোন প্রশ্ন নেই। ১২ বছরের দাম্পত্য জীবন, এই সময়ে এসে ন্যাকামির তো প্রশ্নই আসে না। এখন সবই "দায়িত্ব" আর "কর্তব্যের" খাতিরে করা।

মায়ের রুমে গেলো সে। নিয়ম করে প্রতিদিনই যায় সে। আজকেও তার মা বিছানায় শুয়ে, উঠে বসার সামর্থ্য থাকলেও বসে না সহসা। দুই বছর হলো তার স্ট্রোক হয়েছে, দুই ভাইয়ের বাসায় পালা করে দুই তিন মাস করে রাখা হয় তাকে। এ নিয়ে মায়ের দুঃখের শেষ নেই। ছেলেকে দেখা মাত্র মা বিলাপ জুড়ে দিল, "আজ তর বউ আমারে কী কইসে, জানস?... আমারে দেহার কেউ নাই... আমার দুঃখ কেউ বুঝে না, আমারে ভাগের মা বানায়া রাখসে, মা কি কখনো ভাগ হয়... আমারে সজীবের বাসায় দিয়া আয় তুই, আমি আর এইখানে থাকুম না..."

ছেলে চুপ করে থাকে। তার মা উঠতে বসতে পারে না, খাওয়া দাওয়া বাথরুম করার দরকার পড়লে একে ওকে ডেকে পাড়া মাথায় তোলে, সময় অসময়ে প্রস্রাব পায়খানা করে বিছানা - কাপড় নোংরা করে -- এইসব "বাড়তি" কাজ করতে গিয়ে ওর স্ত্রীর মুখে সারাক্ষণ বিরক্তি লেগে থাকে। সারাদিন বাচ্চা নিয়ে স্কুল, কোচিং, আবৃত্তি সংগঠনে দৌড়াদৌড়ির পর, বাসায় এসে ঘরের কাজ করতে করতে তার জীবন তো অনেক দিন হলো গোল্লায় যাচ্ছে, তার উপর এই উটকো ঝামেলা! ছেলে ব্যস্ত, তার কিছু করার সময় কোথায়? মাঝে কাজের লোক রাখা হয়েছিলো মায়ের দেখাশুনা করার জন্যে, সেখানেও বিপত্তি। হুটহাট ঘর থেকে এটা ওটা গায়েব হয়ে যায়, সেটা নিয়ে গিন্নীর সাথে বুয়ার ঝগড়া --- অন্দরমহলের কাহিনী কাজের বুয়া থেকে অন্য বাড়ির লোকেরা জেনে ফেলেছে , এই নিয়ে আহাজারি --- এইসবের তোড়ে বাধ্য হয়ে গত দুই মাস হলো, বুয়াটাকে বের করে দেয়া হয়েছে। মা সারাদিন ঘরে পড়ে থাকে, ছোট্ট ঘরটা কেমন একটা গুমোট গন্ধে ভরে আছে। এর মধ্যে কিছুক্ষণ থাকার পর ছেলেটার দম বন্ধ হয়ে আসে, মা পারে কী করে?

মায়ের আহাজারি তার পরেও চলতে থাকে, "আমারে ভাগের মা পাইছে... আমারে গাজীপুর দিয়া আয়, আমার কেউ নাই... আমি ওইখানেই ভালো থাকুম... তোরা কেডা আমার জন্যে কী করসস... টেকাটুকা এখনো হাতে রাখলে মাথায় কইরা রাখতি... তুমি মরার আগে ক্যান সব টেকাটুকা দিয়া গেলা গো, অমানুষের বাচ্চাগুলা আমারে ... "

আহাজারি করতে করতে একসময় মা ক্লান্ত হয়ে আসে, ছেলেটা মা কে ঘুম পাড়িয়ে দেয়, তার পর নিজের রুমে আসে।

রাত বাড়তে থাকে। ছেলেটার চোখে অনেক ঘুম। তাও সে জোর করে চোখ মেলে রাখে। হয়তো আজকে গিন্নী সব কাজ সেরে ঘরে ঢুকেই ঘুমের ঘোরে তলিয়ে যাবে না, কিছুক্ষণ জেগে থাকবে... হঠাৎ ছেলেটির মনে পড়ে, এই তো সেদিনও সারাদিনের রোজনামচা ওকে না বলা পর্যন্ত ওর পেটের ভাত হজম হতো না, সেই দিন গুলো কোথায় গেলো?

গিন্নী ঘরে ঢোকে। নাহ, আজকের দিনটিও অন্য দিনগুলোর মতই। বিছানায় গা এলিয়েই ঘুম। বাড়তি একটি কথাও নয়।

ছেলেটি চিৎ হয়ে শুয়ে শুয়ে ভাবে। কালকে আবার আজকের মতই আরেকটি দিন। সকালে ওঠা, উঠে হালকা নাশ্তা করা, তার পর সবার আগে একাকী বেরিয়ে যাওয়া, সারাদিন দৌড়াদৌড়ি, রাতে বাসায় ফেরা। আর এইসব অর্থহীন আকাঙ্ক্ষা নিয়ে সময় কাটানো।

ছেলেটি ভাবে, সে না থাকলে কি এই সময়গুলো একটু হলেও অন্যরকম হতো?

